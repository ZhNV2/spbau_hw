%%%%%%%%%%%%%%%%%%%%%%%%%%%%%%%%%%%%%%%%%
% Short Sectioned Assignment
% LaTeX Template
% Version 1.0 (5/5/12)
%
% This template has been downloaded from:
% http://www.LaTeXTemplates.com
%
% Original author:
% Frits Wenneker (http://www.howtotex.com)
%
% License:
% CC BY-NC-SA 3.0 (http://creativecommons.org/licenses/by-nc-sa/3.0/)
%
%%%%%%%%%%%%%%%%%%%%%%%%%%%%%%%%%%%%%%%%%

%----------------------------------------------------------------------------------------
%	PACKAGES AND OTHER DOCUMENT CONFIGURATIONS
%----------------------------------------------------------------------------------------

\documentclass[paper=a4, fontsize=11pt]{scrartcl} % A4 paper and 11pt font size

\usepackage[T1]{fontenc} % Use 8-bit encoding that has 256 glyphs
\usepackage{fourier} % Use the Adobe Utopia font for the document - comment this line to return to the LaTeX default
\usepackage[english,russian]{babel} % English language/hyphenation
\usepackage{amsmath,amsfonts,amsthm} % Math packages
\usepackage[utf8]{inputenc}

\usepackage{lipsum} % Used for inserting dummy 'Lorem ipsum' text into the template

\usepackage{sectsty} % Allows customizing section commands
\allsectionsfont{\centering \normalfont\scshape} % Make all sections centered, the default font and small caps

\usepackage{fancyhdr} % Custom headers and footers
\pagestyle{fancyplain} % Makes all pages in the document conform to the custom headers and footers
\fancyhead{} % No page header - if you want one, create it in the same way as the footers below
\fancyfoot[L]{} % Empty left footer
\fancyfoot[C]{} % Empty center footer
\fancyfoot[R]{\thepage} % Page numbering for right footer
\renewcommand{\headrulewidth}{0pt} % Remove header underlines
\renewcommand{\footrulewidth}{0pt} % Remove footer underlines
\setlength{\headheight}{13.6pt} % Customize the height of the header

\numberwithin{equation}{section} % Number equations within sections (i.e. 1.1, 1.2, 2.1, 2.2 instead of 1, 2, 3, 4)
\numberwithin{figure}{section} % Number figures within sections (i.e. 1.1, 1.2, 2.1, 2.2 instead of 1, 2, 3, 4)
\numberwithin{table}{section} % Number tables within sections (i.e. 1.1, 1.2, 2.1, 2.2 instead of 1, 2, 3, 4)

\setlength\parindent{0pt} % Removes all indentation from paragraphs - comment this line for an assignment with lots of text

%----------------------------------------------------------------------------------------
%	TITLE SECTION
%----------------------------------------------------------------------------------------

\newcommand{\horrule}[1]{\rule{\linewidth}{#1}} % Create horizontal rule command with 1 argument of height

\title{	
\normalfont \normalsize 
\textsc{Академический университет} \\ [25pt] % Your university, school and/or department name(s)
\horrule{0.5pt} \\[0.4cm] % Thin top horizontal rule
\huge Интерполяционные полиномы \\ % The assignment title
\horrule{2pt} \\[0.5cm] % Thick bottom horizontal rule
}

\author{Николай Жидков} % Your name

\date{\normalsize\today} % Today's date or a custom date

\begin{document}

\maketitle % Print the title

%----------------------------------------------------------------------------------------
%	Краткое описание
%----------------------------------------------------------------------------------------

\section{Краткое описание}

В этом задании я строил интерполяционные полиномы с помощью метода Ньютона.
В качестве сеток использовалась либо равномерная (uniform), либо корни многочлена Чебышева, округленные до ближайших узлов из входной сетки (chebyshev).

%----------------------------------------------------------------------------------------
%	Структура программы
%----------------------------------------------------------------------------------------

\section{Структура программы}

Программа разделена на функции, записанные в файле solve.py. 

\begin{itemize}
	\item $process\_command\_line\_args$, разбор аргументов командной строки:
		\begin{itemize}
		\item Ничего не принимает
		\item Возвращает файл для считывания данных $filename$, флаг полного дебаг вывода $full\_mode$, степень полинома $m$, способ выбора узлов сетки $grid$, флаг построения графика $plot$.
		\end{itemize}
	\item $read(filename)$, чтение данных:
		\begin{itemize}
		\item Принимает имя файла
		\item Возвращает число точек в сетке $n$, массив точек $X$, массив значений в точках $Y$
		\end{itemize}
	\item $uniform(n, a\_, b\_, m)$, рассчитывает равномерную сетку:
		\begin{itemize}
		\item Принимает количество узлов в сетке минус один $n$, начало и конец отрезка $a\_$, $b\_$ и размер той сетки, которую надо построить $m$.
		\item Возращает индексы узлов из начальной сетки, которые мы возьмем в качестве новой подсетки.
		\end{itemize}
	\item $chebyshev(n, a\_, b\_, m)$, рассчитывает сетку из корней многочлена Чебышева:
		\begin{itemize}
		\item Принимает количество узлов в сетке минус один $n$, начало и конец отрезка $a\_$, $b\_$ и размер той сетки, которую надо построить $m$.
		\item Возращает индексы узлов из начальной сетки, которые мы возьмем в качестве новой подсетки.
		\end{itemize}
    \item $deviations(X, Y, P)$, считает отклонения:
        \begin{itemize}
		\item Принимает массив точек $X$, массив значений функции в точках $Y$, массив значений в точках полинома в точках $P$.
		\item Возвращает максимальные абсолютное и относительное отклонение.
		\end{itemize}
	\item $calc\_newton(xes, a, X)$, считает значение интерполяционного многочлена, построенного по методу Ньютона, в указанных точках
		\begin{itemize}
		\item Принимает массив точек для подсчета значения $xes$, коэффициенты многочлена $a$ и массив узлов, по которым он строился $X$.
		\item Возвращает массив значений многочлена в точках $xes$.
		\end{itemize}
	\item $newton(n, X, Y)$, вычисляет коэффициенты интерполяционного многочлена методом Ньютона.
		\begin{itemize}
		\item Принимает степень многочлена $n$, узлы $X$, значения $Y$.
		\item Возвращает значения коэффициентов в разложении $P(x)=\sum a_i (x-x_0)(x-x_1)\dots(x-x_i)$
		\end{itemize}
\end{itemize}


%----------------------------------------------------------------------------------------
%	Структура файлов исходных данных
%----------------------------------------------------------------------------------------

\section{Структура файлов исходных данных}

Во входном файле ожидаются некоторые числа, формат которых описан дальше, при этом наличие пробелов и переводов строк между ними не важен (можно все данные задать в строку через проблел или по одному на строке, это не имеет значения).

Сначала ожидается число $n$ - число узлов.
Дальше идут $n$ чисел - узлы сетки, потом еще $n$ чисел - значения функции в узлах.

Пример входных данных

$3$

$0.01$ $0.02$ $0.03$

$1$ $12$ $3.343$


В результате программе примет функцию, заданную в трех точках $0.01$, $0.02$, $0.03$ со значениями $1$, $12$, $3.343$.
%----------------------------------------------------------------------------------------
%	Примеры вызова из командной строки
%----------------------------------------------------------------------------------------

\section{Примеры вызова из командной строки}

Обязательные флаги (для каждого должно быть обязательно указано какое-то значение):

\begin{itemize}
	\item $--input=$ для указания входного файла (произвольная строка)
    \item $--deg=$ для указания степени полинома (натуральное число)
    \item $--grid=$ для выбора сетки (два варианта - $uniform$ и $chebyshev$)
\end{itemize}

Дополнительные опции (по умолчанию выключены):

\begin{itemize}
	\item $--full$ или $-f$ для вывода подробной информации 
    \item $--plot$ или $-p$ для вывода графика (синим выводится функция, оранжевым полином)
\end{itemize}

Примеры запусков

\begin{itemize}

	\item Строим полином $4$-ой степени по точкам из файла $data$ с помощью равномерной сетки. Дебаг информация не выводится.
	\subitem python3 solve.py $--$input=data $--$deg=5 $--$grid=uniform -p 
   
\end{itemize}

%----------------------------------------------------------------------------------------
%	Численый эксперимент
%----------------------------------------------------------------------------------------

\section{Численный эксперимент}

\subsection{Сравнение полиномов по равномерной сетке}
.

\begin{tabular}{|p{4 cm}|p{4 cm}|p{4 cm}|p{4 cm}|}
\hline
	Критерий анализа & Степень $m=4$ & Степень $m=7$ & Степень $m=14$\\
\hline
	Визуальное сравнение & По структуре довольно похож на данную функцию, но слишком неточен в начале & Большие отклонения в начале, начиная с середины довольно близок к функции & В середине почти везде совпадает с функцией, в начале и конце странные отклонения\\
\hline
	Максимальная абсолютная ошибка & 0.440 & 0.568 & 0.57\\
\hline
	Максимальная относительная ошибка & 0.436 & 0.617 & 0.63\\
\hline
\end{tabular}

\subsection{Построение с помощью корней многочлена Чебышева степени 14}

Максимальная абсолютная ошибка $0.04$. 

Максимальная относительная ошибка $0.04$.

График очень близок к данной функции, в начале и конце имеются незначительные изгибы.

\subsection {Сравнение с МНК}

Перебором вариантов было выяснено, что наимаеньшая абсолютная ошибка достигается при посторении с помощью Чебышева многочлена степени $14$ (результаты выше).

Сравнивая с результатом прошлого задания, можно заметить, что при степени многочлена $14$ МНК давал результат в два раза лучше (там была ошибка порядка $0.02$).

\subsection{Выводы}
\begin{itemize}
\item Использование равномерной сетки не приводит к хорошему результату для данной функции. В целом график довольно похож на функцию, но есть странные пики, природа которых мне не ясна (возможно, это из-за специфики функции).
\item Использование сетки по узлам Чебышева дает куда лучший результат, график очень похож на функцую, отклонения есть по краям, но не очень большие. Для данной функции немного проигрывает методу МНК той же степени ($0.04$ вместо $0.02$).
\end{itemize}


\end{document}



