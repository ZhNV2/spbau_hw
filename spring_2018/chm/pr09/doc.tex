%%%%%%%%%%%%%%%%%%%%%%%%%%%%%%%%%%%%%%%%%
% Short Sectioned Assignment
% LaTeX Template
% Version 1.0 (5/5/12)
%
% This template has been downloaded from:
% http://www.LaTeXTemplates.com
%
% Original author:
% Frits Wenneker (http://www.howtotex.com)
%
% License:
% CC BY-NC-SA 3.0 (http://creativecommons.org/licenses/by-nc-sa/3.0/)
%
%%%%%%%%%%%%%%%%%%%%%%%%%%%%%%%%%%%%%%%%%

%----------------------------------------------------------------------------------------
%	PACKAGES AND OTHER DOCUMENT CONFIGURATIONS
%----------------------------------------------------------------------------------------

\documentclass[paper=a4, fontsize=11pt]{scrartcl} % A4 paper and 11pt font size

\usepackage[T1]{fontenc} % Use 8-bit encoding that has 256 glyphs
\usepackage{fourier} % Use the Adobe Utopia font for the document - comment this line to return to the LaTeX default
\usepackage[english,russian]{babel} % English language/hyphenation
\usepackage{amsmath,amsfonts,amsthm} % Math packages
\usepackage[utf8]{inputenc}

\usepackage{lipsum} % Used for inserting dummy 'Lorem ipsum' text into the template

\usepackage{sectsty} % Allows customizing section commands
\allsectionsfont{\centering \normalfont\scshape} % Make all sections centered, the default font and small caps

\usepackage{fancyhdr} % Custom headers and footers
\pagestyle{fancyplain} % Makes all pages in the document conform to the custom headers and footers
\fancyhead{} % No page header - if you want one, create it in the same way as the footers below
\fancyfoot[L]{} % Empty left footer
\fancyfoot[C]{} % Empty center footer
\fancyfoot[R]{\thepage} % Page numbering for right footer
\renewcommand{\headrulewidth}{0pt} % Remove header underlines
\renewcommand{\footrulewidth}{0pt} % Remove footer underlines
\setlength{\headheight}{13.6pt} % Customize the height of the header

\numberwithin{equation}{section} % Number equations within sections (i.e. 1.1, 1.2, 2.1, 2.2 instead of 1, 2, 3, 4)
\numberwithin{figure}{section} % Number figures within sections (i.e. 1.1, 1.2, 2.1, 2.2 instead of 1, 2, 3, 4)
\numberwithin{table}{section} % Number tables within sections (i.e. 1.1, 1.2, 2.1, 2.2 instead of 1, 2, 3, 4)

\setlength\parindent{0pt} % Removes all indentation from paragraphs - comment this line for an assignment with lots of text

%----------------------------------------------------------------------------------------
%	TITLE SECTION
%----------------------------------------------------------------------------------------

\newcommand{\horrule}[1]{\rule{\linewidth}{#1}} % Create horizontal rule command with 1 argument of height

\title{	
\normalfont \normalsize 
\textsc{Академический университет} \\ [25pt] % Your university, school and/or department name(s)
\horrule{0.5pt} \\[0.4cm] % Thin top horizontal rule
\huge Решение дифференциальных уравнений  \\ % The assignment title
\horrule{2pt} \\[0.5cm] % Thick bottom horizontal rule
}

\author{Николай Жидков} % Your name

\date{\normalsize\today} % Today's date or a custom date

\begin{document}

\maketitle % Print the title

%----------------------------------------------------------------------------------------
%	Предисловия
%----------------------------------------------------------------------------------------

\section{Структура программы}

Во-первых, с уравнением 2-го порядка из условия что-то не так, потому что функция $y(x)$ не является решением это уравнения и под начальные условия не подходит. Поэтому я поменял условия на $y(1)=-1, y(x)=-\sqrt{x}$, так все сходится.

Во-вторых, я использовал в коде в некоторый момент массивы из модуля $numpy$, потому что их можно складывать друг с другом и домножать на коэффициенты. Мне это было нужно, чтобы задачи из обоих пунктов решать одинковыми функциями.

%----------------------------------------------------------------------------------------
%	Структура программы
%----------------------------------------------------------------------------------------

\section{Структура программы}

Программа разделена на функции, записанные в файле solve.py. Содержательные функции: 

\begin{itemize}
	\item $evaluate_y(a, b, n, y0, f)$, выполняет проход метода Рунге-Кутта 3-го порядка:
		\begin{itemize}
		\item Принимает границы отрезка $a, b$, количество отрезков $n$ (количество узлов минус один), начальное значение $y0$ и функцию $f$.
		\item Возвращает посчитанный массив $Y$ (в первом пункте это массив значений в точках, во втором пункте это массив пар (значение, производная)).
		\end{itemize}
	\item $solve(a, b, f, y0, start\_n, eps, max\_n))$, запускает Рунге-Кутта с нужными параметрами:
		\begin{itemize}
		\item Принимает границы отрезка $a, b$, начальное значение $y0$ и функцию $f$, начальное количество отрезков $start\_n$, граничное значение для оценки Рунге $eps$ и максимальное число отрезков $max\_n$.
		\item Вычисляет массив значений в точках, каждый раз увеличивая сетку вдвое. Останавливается, когда $n$ стало равно $max\_n$ или когда оценки Рунге на двух последовательных прогонах стала меньше $eps$ для всех значений.
		\end{itemize}
    \item $deviations*(Y, Y\_t, start\_n, *)$, выводит информацию по вычислениям:
        \begin{itemize}
		\item Принимает вычисленные значения $Y$, реальные значения $Y\_t$ и количство контрольных точек $start\_n$.
		\item Печатает информацию об отклонениях в вычислениях.
		\end{itemize}
    \item $process\_command\_line\_args()$, считывет аргументы командой строки:
        \begin{itemize}
		\item Ничего не принимает
		\item Возвращает метод для запуска $run$ и ее параметр $param$.
		\end{itemize}
    \item $test*(a, b, y0, start_n, eps, y_t, f)$, запускает вычисления для данного уравнения первого порядка:
        \begin{itemize}
		\item Принимает границы отрезка $a, b$, начальное значение $y0$ и функцию $f$, начальное количество отрезков $start\_n$, граничное значение для оценки Рунге $eps$ и функцию ответа $y\_t$
		\item Проводит вычисления, печатает ответ.
		\end{itemize}
\end{itemize}


%----------------------------------------------------------------------------------------
%	Структура файлов исходных данных
%----------------------------------------------------------------------------------------

\section{Структура файлов исходных данных}

В данной задаче файлы не используются.

%----------------------------------------------------------------------------------------
%	Примеры вызова из командной строки
%----------------------------------------------------------------------------------------

\section{Примеры вызова из командной строки}

Есть 4 режима работы для решения задач из условия

\begin{itemize}
	\item Решаем уравнение первого порядка, ожидая порогового значения 0.001
	\subitem $python3$ $solve.py$ $--run=1$ $--param=0.001$
    \item Решаем уравнение второго порядка, деля интервал на 20 отрезков
	\subitem $python3$ $solve.py$ $--run=2$ $--param=20$
    \item Решаем уравнение второго порядка, вносим возмущение в $y(1)$ в размере 5 процентов (количество отрезков всегда 10).
	\subitem $python3$ $solve.py$ $--run=2.1$ $--param=5$
    \item Решаем уравнение второго порядка, вносим возмущение в $y'(1)$ в размере 5 процентов (количество отрезков всегда 10).
	\subitem $python3$ $solve.py$ $--run=2.2$ $--param=5$
    
\end{itemize}


%----------------------------------------------------------------------------------------
%	Численый эксперимент
%----------------------------------------------------------------------------------------

\section{Численный эксперимент}

\subsection{Уравнение первого порядка}

\begin{tabular}{|p{4 cm}|p{4 cm}|p{4 cm}|}
\hline
	eps & количество отрезков & максимальное отклонение интеграла в процентах от eps \\
\hline 
	0.001 & 20 & 6\\
\hline 
	0.0001 & 40 & 8\\
\hline
	0.00001 & 80 & 11\\
\hline
\end{tabular}

Как видно из таблицы, с уменьшением $eps$ на порядок, количество проходов увеличивается на 1 (что то же самое, число отрезков увеличивается вдвое). Точность, как легко видеть, падает также на порядок, что вполне ожидаемо.

Довольно неожиданным только является факт, что при том, что максимальное отклонение мы разрешаем $eps$, максимальное отклонение в контрольных узлах дает лишь $10$ процентов от него.

\subsection{Уравнение второго порядка, вычисление значений}

\begin{tabular}{|p{4 cm}|p{4 cm}|}
\hline
	max n & максимальное отклонение интеграла\\
\hline 
	10 & 0.00000243 \\
\hline 
	20 & 0.00000029 \\
\hline
	40 & 0.00000004 \\
\hline
\end{tabular}

Как видно, с увеличением числа узлов в два раза, максимальное отклонение падает на порядок.

\subsection{Уравнение второго порядка, возмущения}

В таблице записаны максимальные отклонения

\begin{tabular}{|p{4 cm}|p{4 cm}|p{4 cm}|}
\hline
	Процент & В значение &  В производную \\
\hline 
	3 & 0.03 & 0.00000243\\
\hline 
	4 & 0.04 & 0.00463403\\
\hline
	5 & 0.05 & 0.01854162\\
\hline
\end{tabular}

В левом столбце стоят значения процентов, так как при изменении начального значения на $p$ процентов, потом все значения примерно сохраняют это изменение.
В целом видно, что производная гораздо меньше влияет на результат, что довольно логично, так изменение начального значение оказывает непосредственное влияние на дальнейшие вычисления.



\end{document}



